\chapter{Nutzung des Templates}\label{kap_muster}

Die folgenden Inhalte demonstrieren die Nutzung dieses Latex-Templates.

\section{Unterkapitel 1}\label{kap_unterkapitel1}

Hier entsteht ein erstes Unterkapitel.

Im Text können Grafiken verwendet und bei Bedarf beschriftet werden.

\begin{figure}[h!]
\centering
\includegraphics[width=.6\linewidth]{decorative/University_Logo.png}
\caption{Wunderschöne Darstellung, Quelle: \cite{THKoeln2024}}
\label{fig_my-beautiful-image}
\end{figure}

Manche Daten lassen sich besser in einer Tabelle darstellen.

\begin{table}[h!]
\centering
\begin{tabular}{|c c c|} 
\hline
Jahreszeit & Erdbeeren & Maronen \\
\hline
Sommer & super lecker & lecker \\ 
Herbst & lecker & super lecker \\
\hline
\end{tabular}
\caption{Tabelle mit Lebensmittelratschlägen}
\label{tab_food}
\end{table}

Wenn nötig können zudem Code-Ausschnitte eingebunden werden.

\begin{lstlisting}[breaklines, captionpos=b, caption={Mein wunderschöner Code}, label={lst_listing1}]
export const myUsefulConstant = "I'm no use at all!";
\end{lstlisting}

Vorab definierte Abkürzungen können per Hand zum Abkürzungsverzeichnis hinzugefügt werden und sind im Text nicht sichtbar.

\glsadd{exmp1}

In der Regel werden sie jedoch im Text verwendet und dadurch automatisch zum Verzeichnis hinzugefügt.
Sie können dabei unterschiedliche Formen annehmen.

Voll: \acrfull{exmp2}

Lang: \acrlong{exmp2}

Kurz: \acrshort{exmp2}

Quellenangabe erfolgen durch Referenz auf einen Eintrag in der Bibliographie.

\cite{Wright2010}
    
Für wörtliche Zitate sind Anführungszeichen zu setzen.

\textquote{Once upon a time...}

Elemente, die über ein Label verfügen, können referenziert werden. Dazu zählen z. B.

das Kapitel \ref{kap_muster},

die Abbildung \ref{fig_my-beautiful-image},

die Tabelle \ref{tab_food}

und das Listing \ref{lst_listing1}

\section{Unterkapitel 2} \label{kap_unterkapitel2}
\lipsum[1-5]

\subsection{Unterunterkapitel} \label{kap_unterunterkapitel}
\lipsum[1-5]

\subsubsection{Unterunterunterkapitel} \label{kap_unterunterunterkapitel}
\lipsum[1-5]
