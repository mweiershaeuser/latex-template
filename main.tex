\documentclass[a4paper,12pt,oneside]{report}

% METADATEN
\newcommand{\TITLE}{<Der ultimative Arbeitstitel dieses Meisterwerkes>}
\newcommand{\AUTHOR}{<Vorname Nachname>}
\newcommand{\DATE}{\today}
\title{\TITLE}
\author{\AUTHOR}
\date{\DATE}

% SPRACHE
\usepackage[utf8]{inputenc}
\usepackage[T1]{fontenc}
\usepackage[ngerman]{babel} % Support für die deutsche Sprache
\usepackage{hyphenat} % Worttrennung

% LAYOUT
\usepackage{geometry}
\geometry{
    left=3cm,
    right=2cm,
    top=1cm,
    bottom=1cm,
    includeheadfoot,
    headheight=1cm,
    headsep=1cm,
    footskip=1cm,
}
\usepackage{fancyhdr}
\pagestyle{fancy}
\fancypagestyle{fancy}{ % Header/Footer Style für den Hauptteil
    \fancyhf{}
    \fancyhf[HC]{\nouppercase{\leftmark}} % Header, Center -> aktuellen Kapitelnamen
    \fancyhf[FC]{\thepage} % Footer, Center -> Seitenzahl
    \renewcommand{\headrulewidth}{1pt} % Headerlinie
}
\fancypagestyle{plain}{} % Überschreibe plain mit fancy, da erste Seite eines Chapters immer plain annimmt
\fancypagestyle{frontmatter}{
    \fancyhf{}
    \fancyhf[FC]{\thepage}
    \renewcommand{\headrulewidth}{0pt}
}
\fancypagestyle{backmatter}{
    \fancyhf{}
    \fancyhf[FC]{\thepage}
    \renewcommand{\headrulewidth}{0pt}
}
\usepackage{titlesec}
\titleformat{\chapter}{\huge\normalfont\bfseries}{\thechapter}{1em}{} % Kapitel-Präfix
\titlespacing{\chapter}{0em}{0em}{1em} % Kapitelnamen Spacing
\renewcommand{\chaptermark}[1]{\markboth{\thechapter\ #1}{}} % Kapitelnamen im Header

% TEXT
\usepackage[sfdefault]{roboto}
\usepackage{parskip} % Absatzabstand
\usepackage{setspace} % Zeilenabstand
\onehalfspacing % Zeilenabstand

\usepackage[normalem]{ulem} % Underlining
\usepackage{hyperref} % Links und PDF-Einstellungen
\hypersetup{
    hidelinks,
    hypertexnames=false,
    plainpages=false,
    bookmarksopen=true,
    bookmarksnumbered=true,
    pdfview=FitH,
    pdfstartview=FitH,
    pdftitle={\TITLE},
    pdfauthor={\AUTHOR}
}

% BILDER
\usepackage{graphicx}
\usepackage[export]{adjustbox} % Grafikoptionen
\usepackage{caption}

% LISTINGS
\usepackage{listings}

% ABKÜRZUNGSVERZEICHNIS
\usepackage[acronym, toc, automake]{glossaries}
\makeglossaries
\loadglsentries[acronym]{acronyms}

% LITERATUR
\usepackage[style=alphabetic]{biblatex}
\addbibresource{literature.bib}
\usepackage{csquotes} % Anführungszeichen bei Zitaten

% ANHANG
\usepackage[page,toc,titletoc]{appendix}
\renewcommand{\appendixpagename}{Anhang}
\renewcommand{\appendixtocname}{Anhang}

% SONSTIGES
\usepackage{lipsum} % Fülltext, bei Nutzung des Templates entfernen

%%%%%%%%%% BEGINN DES DOKUMENTS %%%%%%%%%%

\begin{document}

\pagestyle{frontmatter}
\pagenumbering{Roman}

\begin{titlepage}
\begin{singlespace}

\includegraphics[width=0.3\textwidth, right]{decorative/University_Logo.png}

\vspace{6cm}

{\huge \bfseries \TITLE}

\vspace{0.5cm}

<Masterarbeit> zur Erlangung des akademischen Grades <Master of Science (M.Sc.)>

\vspace{2cm}

ausgearbeitet von:

\AUTHOR

\vspace{1cm}

vorgelegt an der:

<Hochschule>\\
Campus <Standort>\\
Fakultät <Fakultät>

im Studiengang <Studiengang>\\
mit dem Schwerpunkt <Schwerpunkt>

\vspace{1cm}

Erstprüfer: <Prof. Dr. Vorname Nachname>\\
Zweitprüfer: <Prof. Dr. Vorname Nachname>

\vspace{1cm}

<Ort>, \DATE

\end{singlespace}
\end{titlepage}


\setcounter{page}{2}

\input{formalities/Abstract.tex}
\input{formalities/Inhaltsverzeichnis.tex}
\newpage
\phantomsection
\addcontentsline{toc}{chapter}{Abbildungsverzeichnis}
\listoffigures

\newpage
\phantomsection
\addcontentsline{toc}{chapter}{Tabellenverzeichnis}
\listoftables
\input{formalities/Listingverzeichnis.tex}
\newpage
\printglossary[type=\acronymtype, title=Abkürzungsverzeichnis, toctitle=Abkürzungsverzeichnis]


\newpage
\pagestyle{fancy}
\pagenumbering{arabic}

\chapter{Einleitung}

\lipsum[1-3]

\chapter{Nutzung des Templates}\label{kap_muster}

Die folgenden Inhalte demonstrieren die Nutzung dieses Latex-Templates.

\section{Unterkapitel 1}\label{kap_unterkapitel1}

Hier entsteht ein erstes Unterkapitel.

Im Text können Grafiken verwendet und bei Bedarf beschriftet werden.

\begin{figure}[h!]
\centering
\includegraphics[width=.6\linewidth]{decorative/University_Logo.png}
\caption{Wunderschöne Darstellung, Quelle: \cite{THKoeln2024}}
\label{fig_my-beautiful-image}
\end{figure}

Manche Daten lassen sich besser in einer Tabelle darstellen.

\begin{table}[h!]
\centering
\begin{tabular}{|c c c|} 
\hline
Jahreszeit & Erdbeeren & Maronen \\
\hline
Sommer & super lecker & lecker \\ 
Herbst & lecker & super lecker \\
\hline
\end{tabular}
\caption{Tabelle mit Lebensmittelratschlägen}
\label{tab_food}
\end{table}

Wenn nötig können zudem Code-Ausschnitte eingebunden werden.

\begin{lstlisting}[breaklines, captionpos=b, caption={Mein wunderschöner Code}, label={lst_listing1}]
export const myUsefulConstant = "I'm no use at all!";
\end{lstlisting}

Vorab definierte Abkürzungen können per Hand zum Abkürzungsverzeichnis hinzugefügt werden und sind im Text nicht sichtbar.

\glsadd{exmp1}

In der Regel werden sie jedoch im Text verwendet und dadurch automatisch zum Verzeichnis hinzugefügt.
Sie können dabei unterschiedliche Formen annehmen.

Voll: \acrfull{exmp2}

Lang: \acrlong{exmp2}

Kurz: \acrshort{exmp2}

Quellenangabe erfolgen durch Referenz auf einen Eintrag in der Bibliographie.

\cite{Wright2010}
    
Für wörtliche Zitate sind Anführungszeichen zu setzen.

\textquote{Once upon a time...}

Elemente, die über ein Label verfügen, können referenziert werden. Dazu zählen z. B.

das Kapitel \ref{kap_muster},

die Abbildung \ref{fig_my-beautiful-image},

die Tabelle \ref{tab_food}

und das Listing \ref{lst_listing1}

\section{Unterkapitel 2} \label{kap_unterkapitel2}
\lipsum[1-5]

\subsection{Unterunterkapitel} \label{kap_unterunterkapitel}
\lipsum[1-5]

\subsubsection{Unterunterunterkapitel} \label{kap_unterunterunterkapitel}
\lipsum[1-5]

\chapter{Schlussbetrachtung}

\lipsum[1-3]


\newpage
\pagestyle{backmatter}

\chapter*{Literaturverzeichnis}
\addcontentsline{toc}{chapter}{Literaturverzeichnis}
\printbibliography[heading=none]

\begin{appendices}

\chapter{Erster toller Anhang}

Erster toller Anhang

\chapter{Zweiter toller Anhang}

Zweiter toller Anhang

\end{appendices}

\chapter*{Erklärung über die selbständige\\Abfassung der Arbeit}
\addcontentsline{toc}{chapter}{Erklärung über die selbständige Abfassung der Arbeit}
Ich versichere, die von mir vorgelegte Arbeit selbständig verfasst zu haben.
Alle Stellen, die wörtlich oder sinngemäß aus veröffentlichten oder nicht veröffentlichten Arbeiten anderer entnommen sind, habe ich als entnommen kenntlich gemacht.\\
Sämtliche Quellen und Hilfsmittel, die ich für die Arbeit benutzt habe, sind angegeben. Die Arbeit hat mit gleichem Inhalt bzw. in wesentlichen Teilen noch keiner anderen Prüfungsbehörde vorgelegen.
\\\\\\\\
\begin{tabular}{cp{7cm}}
& \\
& \\ \hline
\small (Ort, Datum, Unterschrift) & \normalsize \\
\end{tabular}


\end{document}
